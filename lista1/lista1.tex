\documentclass{article}

% Language setting
% Replace `english' with e.g. `spanish' to change the document language
\usepackage[portuguese]{babel}

% Set page size and margins
% Replace `letterpaper' with`a4paper' for UK/EU standard size
\usepackage[letterpaper,top=2cm,bottom=2cm,left=3cm,right=3cm,marginparwidth=1.75cm]{geometry}

% Useful packages
\usepackage{amsmath}
\usepackage{amssymb}
\usepackage{setspace}
\usepackage{graphicx}
\usepackage[colorlinks=true, allcolors=blue]{hyperref}
\DeclareMathSizes{12}{30}{16}{12}

\title{\textbf{Lista 1 - MAC0425 Inteligência Artificial}}
\author{Daniel Angelo Esteves Lawand 10297693}

\begin{document}
\maketitle
\onehalfspacing
\section*{Exercício 1}

Um agente inteligente é aquele que percebe e age em um ambiente. Dito isso, podemos dividir o processo da tomada de decisão do agente inteligente em duas partes: perceber (pensar) o ambiente e agir no ambiente. Assim como, um agente inteligente pode ser racional ou ser humano.\\
\\
Na dimensão de agentes inteligentes que pensam como humanos, o foco não está se a solução do problema está correta, mas sim se os passos de execução do problema eram similares aos passos de raciocínio humano.\\
\\
Na dimensão de agentes que agem como humanos, o foco está na construção de um agente que responde ao ambiente de maneira semelhante aos humanos.\\
\\
Na dimensão de agentes que pensam racionalmente, o foco está na construção de agentes que façam o uso de lógica (argumentos bem estruturados que produzem conclusões corretas quando dadas uma premissas corretas), ou façam o uso de probabilidade quando a informação é incerta.\\
\\
E na dimensão de agentes que agem racionalmente, o foco está em perceber o ambiente e executar ações para atingir o melhor resultado ou, quando houver incerteza, o melhor resultado esperado.\\
\\
Sabendo o conjunto de percepções P, o conjunto de ações A, conjunto de estados do ambiente, o programa do agente é f: P -> A, e que a função desempenho é g: S*xA* -> [-inf, inf]. Sabemos que um agente é inteligente se f otimiza g, caso contrário não é.

\section*{Exercício 2}
Definição I - pensar racional\\
\\
Definição II - pensar humano\\
\\
Definição III - agir humano\\

\section*{Exercício 3}
- Dificuldade em listar todas possíveis ações dado uma percepção\\
- Dificuldade em listar todas as possíveis percepções\\
- Dificuldade em armazenar todas as possbilidades (anteriormente comentadas)\\
- Em um ambiente não determinístico, há a dificuldade de tomar uma decisão, portanto haveria uma probabilidade atrelada a cada possível decisão.\\
- Dificuldade no cálculo da probabilidade para possível decisão em um ambiente não determinístico.\\
- Dificuldade de considerar a consequência futura das ações do agente.\\

\section*{Exercício 4}

Parte I\\
1\\
2\\
3\\
4\\
5\\
6\\

Parte II\\
1\\
2\\
4\\
8\\
...\\

Não é possível encontrá-lo, já que a sub-árvore enraizada no nó 8 é de profundidade infinita.\\


Parte III\\
1\\
1\\
2\\
3\\
1\\
2\\
4\\
5\\
3\\
6\\

\section*{Exercício 5}
\subsection{A}

S = {
    {{MMMCCC}, },
    {},
    {},
    {},
    {},
    {},
    {},
    }


<S, A, T, g, M, s_0>

No caso do exercício:
          Margem 1   Margem 2
s_0 = [MMMCCC,   VAZIO]
M  = [VAZIO,       MMMCCC]

       Quem pode ir para o barco
A = [C, M, CM, CC, MM]
\end{document}